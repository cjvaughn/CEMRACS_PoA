\documentclass[]{article}

%opening
\title{Response to Referee}
\author{}

\begin{document}

\maketitle

We would like to thank the reviewer for their remarks. We have addressed each of their points as follows:\\

\hspace{-7mm} \textbf{Main Points:}
\begin{enumerate}
	\item This was a good suggestion. We have added equations (3)-(7) on page 7, and equations (9)-(13) on page 8, and we refer to these equations often. In the same vein we have added equations (26)-(29) on pages 13-14, which are used frequently in Propositions 3-7.
	\item Since the Hamiltonian depends on the scalars $\bar{\mu}$ and $\bar{\nu}$, instead of the full measures, the derivative in the measure argument greatly simplifies. On page 8 line 6, we refer to section 4 in [5], where they write what this condition should be in the linear quadratic case. On page 8 lines 7 and 8, the derivative is a standard derivative and does not need explaining further. There was a typo in the previous version that may have caused confusion, where we mistakenly wrote $\partial_{\mu}$ instead of $\partial_{\bar{\mu}}$.
	\item 
	\item Yes, it is correct that there is a square in the definitions of $C^w$ and $D^w$ that doesn't appear in the definitions of $C^u$ and $C^w$. This was overlooked by the authors. If we were to rewrite Corollary 3 with the correct condition to have $(A^u,B^u,C^u,D^u)=(A^w,B^w,C^w,D^w)$, it would not be substantially different from Corollary 1 and thus, we dropped Corollary 3.
	\item
	\item
	\item We added a sentence at the beginning of section 2.4 on page 27 stating that we use a simple rectangular integration rule to numerically compute integrals.
\end{enumerate}

\hspace{-7mm} \textbf{Minor Points:}
\begin{enumerate}
	\item We agree that this notation was awkward. We have changed all of the function definitions to the more standard notation.
	\item We changed `its' to `their' to have consistent pronoun usage.
	\item We added `game'.
	\item We have used the pronoun `their' throughout the text, so we will continue to use it here.
	\item The missing parenthesis have been added.
	\item We agree. Now page 6 line 10 reads `According to the Pontryagin stochastic maximum principle, a sufficient condition for optimality is...'
	\item Yes, this was a typo. On page 6 line 13, we have replaced $x$ by $X_t$. At this point, the fixed point is not addressed, and the control is the best response when the mean field is fixed. So we should not yet replace $\bar{\mu}$ and $\bar{\nu}$.
	\item On page 6 line -6, we have added: `Note $c^{MFG}(t)=a^{MFG}(t)+b^{MFG}(t)$.' We also added `Note $c^{MKV}(t)=a^{MKV}(t)+b^{MKV}(t)$.' on page 8 line -12.
	\item The extra parenthesis has been removed.
	\item We have added equation (6) and refer to it in the proof of Corollary 2, page 13 line 2. And yes, it should be $\bar{x}_t^{MFG}$.
	\item We changed `tends' to `tend'.
	\item In the proof of Proposition 3, page 15 line -6, we replaced the limit of $v^r_t$ with our new notation $v^{r \to \infty}_t$.
	\item Yes, this was a typo. These values were indeed after taking the limit. The limits in a slightly different presentation now appear in the proof of Proposition 5 Case 2, page 19 line 12.
\end{enumerate}

\hspace{-7mm} In addition to the suggested revisions, we have also made the following minor modifications:
\begin{enumerate}
	\item 
\end{enumerate}

Finally, we would like to acknowledge that the length of our paper has increased from 26 pages at submission to 33 pages for the final version. The issues in our proofs raised by the reviewer required us to write many more details, and in some cases, new approaches altogether. We feel that the increase in length was necessary to address all of the issues pointed out by the reviewer. (ToDo: Is this comment necessary?)

\end{document}
